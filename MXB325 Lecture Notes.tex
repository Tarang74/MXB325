%!TEX TS-program = xelatex
%!TEX options = -aux-directory=Debug -shell-escape -file-line-error -interaction=nonstopmode -halt-on-error -synctex=1 "%DOC%"
\documentclass{article}
\input{LaTeX-Submodule/template.tex}

% Additional packages & macros
\theoremstyle{definition}
\newtheorem{example}{Example}[section]

% Header and footer
\newcommand{\unitName}{Differential Equations and Modelling 2}
\newcommand{\unitTime}{Semester 2, 2024}
\newcommand{\unitCoordinator}{Professor Scott McCue}
\newcommand{\documentAuthors}{Tarang Janawalkar}

\fancyhead[L]{\unitName}
\fancyhead[R]{\leftmark}
\fancyfoot[C]{\thepage}

% Copyright
\usepackage[
  type={CC},
  modifier={by-nc-sa},
  version={4.0},
  imagewidth={5em},
  hyphenation={raggedright}
]{doclicense}

\date{}

\begin{document}
%
\begin{titlepage}
    \vspace*{\fill}
    \begin{center}
        \LARGE{\textbf{\unitName}} \\[0.1in]
        \normalsize{\unitTime} \\[0.2in]
        \normalsize\textit{\unitCoordinator} \\[0.2in]
        \documentAuthors
    \end{center}
    \vspace*{\fill}
    \doclicenseThis
    \thispagestyle{empty}
\end{titlepage}
\newpage
%
\tableofcontents
\newpage
%
\part{Symmetry Methods}
\section{Symmetry Transformations}
Consider a partial differential equation for \(u\left( x,\: t \right)\)
whose domain lies in \(\R^2\). Such a problem typically does not have
any natural length or time scales associated with it's fundamental
solution. Thus, let us transform the independent and dependent
variables through the mapping \(\left( x,\: t,\: u \right) \mapsto
\left( X,\: T,\: U \right)\) where \(X = X\left( x,\: t,\: u \right)\),
\(T = T\left( x,\: t,\: u \right)\), and \(U = U\left( x,\: t,\: u
\right)\), such that the PDE is invariant under this transformation,
that is, the transformed PDE in \(U\) has the same form as the original
PDE in \(u\).

The choice of mapping may consist of dilations and translations of the
independent and dependent variables by some constant factors, where
each constant is enforced by the invariance condition. To do this, we
must ensure that the function \(U\left( x,\: t,\: u\left( X,\: T
\right) \right)\) satisfies the transformed PDE. When this is the case,
the PDE is said to have a \textit{symmetry transformation}.

The goal of these symmetry methods is to find an appropriate
transformation that simplies the PDE, often reducing it to an ODE.
\subsection{Dilation Symmetry}
A dilation symmetry is a transformation of the form
\begin{equation*}
    X = a x, \quad T = a^\beta t, \quad U = a^\gamma u,
\end{equation*}
where \(a\), \(\beta\), and \(\gamma\) are constants. Note that we do
not express the transformation for \(x\) as \(X = a^\alpha x\) since it
does not provide any additional information.
\begin{example}
    Consider the transport equation
    \begin{equation*}
        \pdv{u\left( x,\: t \right)}{t} + c \pdv{u\left( x,\: t \right)}{x} = 0
    \end{equation*}
    and the mapping
    \begin{equation*}
        X = a x, \quad T = a^\beta t, \quad U = u.
    \end{equation*}
    After substituting the transformed solution into the PDE, we find
    \begin{align*}
        \pdv{u\left( X,\: T \right)}{t} + c \pdv{u\left( X,\: T \right)}{x}                       & = 0  \\
        \pdv{u\left( X,\: T \right)}{T} \odv{T}{t} + c \pdv{u\left( X,\: T \right)}{X} \odv{X}{x} & = 0  \\
        a^\beta \pdv{u\left( X,\: T \right)}{T} + a c \pdv{u\left( X,\: T \right)}{X}             & = 0.
    \end{align*}
    For this PDE to be invariant under a dilation transformation, we
    must factor out the constant \(a\) from the equation. This implies
    that:
    \begin{equation*}
        a^\beta = a \implies \beta = 1,
    \end{equation*}
    so that
    \begin{equation*}
        \pdv{u\left( X,\: T \right)}{T} + c \pdv{u\left( X,\: T \right)}{X} = 0.
    \end{equation*}
    Here we have shown that the transport equation is invariant under
    the above dilation transformation when \(\beta = 1\).
\end{example}
\begin{example}
    Consider the nonlinear PDE
    \begin{equation*}
        \pdv{u\left( x,\: y \right)}{x} + y^2 \pdv{u\left( x,\: y \right)}{y} = 0
    \end{equation*}
    and the mapping
    \begin{equation*}
        X = a x, \quad Y = a^\beta y, \quad U = u.
    \end{equation*}
    By substituting the transformed solution into the PDE, we find
    \begin{align*}
        \pdv{u\left( X,\: Y \right)}{x} + y^2 \pdv{u\left( X,\: Y \right)}{y}                                    & = 0  \\
        \pdv{u\left( X,\: Y \right)}{X} \odv{X}{x} + y^2 \pdv{u\left( X,\: Y \right)}{Y} \odv{Y}{y}              & = 0  \\
        a \pdv{u\left( X,\: Y \right)}{X} + a^\beta y^2 \pdv{u\left( X,\: Y \right)}{Y}                          & = 0  \\
        a \pdv{u\left( X,\: Y \right)}{X} + a^\beta \left( Y / a^\beta \right)^2 \pdv{u\left( X,\: Y \right)}{Y} & = 0  \\
        a \pdv{u\left( X,\: Y \right)}{X} + a^{-\beta} Y^2 \pdv{u\left( X,\: Y \right)}{Y}                       & = 0.
    \end{align*}
    For this PDE to be invariant under a dilation transformation, we
    must again factor out \(a\) from the equation so that:
    \begin{equation*}
        a = a^{-\beta} \implies 1 = -\beta \implies \beta = -1,
    \end{equation*}
    and
    \begin{equation*}
        \pdv{u\left( X,\: Y \right)}{X} + Y^2 \pdv{u\left( X,\: Y \right)}{Y} = 0.
    \end{equation*}
    Here we have shown that this nonlinear PDE is invariant under the
    above dilation transformation when \(\beta = -1\).
\end{example}
\begin{example}
    Consider the nonlinear convection diffusion equation
    \begin{equation*}
        \pdv{u\left( x,\: t \right)}{t} = u\left( x,\: t \right) \pdv[order=2]{u\left( x,\: t \right)}{x} - \pdv{u\left( x,\: t \right)}{x}
    \end{equation*}
    and the mapping
    \begin{equation*}
        X = a x, \quad T = a^\beta t, \quad U = a^\gamma u.
    \end{equation*}
    We will once again substitute the transformed solution into the PDE
    to find
    \begin{align*}
        a^\gamma \pdv{u\left( X,\: T \right)}{t}            & = a^{2\gamma} u\left( X,\: T \right) \pdv[order=2]{u\left( X,\: T \right)}{x} - a^\gamma \pdv{u\left( X,\: T \right)}{x}                                       \\
        a^\gamma \pdv{u\left( X,\: T \right)}{T} \odv{T}{t} & = a^{2\gamma} u\left( X,\: T \right) \pdv*{\left[ \pdv{u\left( X,\: T \right)}{X} \odv{X}{x} \right]}{x} - a^\gamma \pdv{u\left( X,\: T \right)}{X} \odv{X}{x} \\
        a^{\beta + \gamma} \pdv{u\left( X,\: T \right)}{T}  & = a^{1 + 2\gamma} u\left( X,\: T \right) \pdv*{\left[ \odv{X}{x} \pdv{u\left( X,\: T \right)}{X} \right]}{X} - a^{1 + \gamma} \pdv{u\left( X,\: T \right)}{X}  \\
        a^{\beta + \gamma} \pdv{u\left( X,\: T \right)}{T}  & = a^{2 + 2\gamma} u\left( X,\: T \right) \pdv[order=2]{u\left( X,\: T \right)}{X} - a^{1 + \gamma} \pdv{u\left( X,\: T \right)}{X}.
    \end{align*}
    For this PDE to be invariant under a dilation transformation, we
    must factor out \(a\) from the equation so that:
    \begin{equation*}
        a^{\beta + \gamma} = a^{2 + 2\gamma} = a^{1 + \gamma} \implies \beta + \gamma = 2 + 2\gamma = 1 + \gamma \implies \beta = 1, \gamma = -1,
    \end{equation*}
    and
    \begin{equation*}
        \pdv{u\left( X,\: T \right)}{T} = u\left( X,\: T \right) \pdv[order=2]{u\left( X,\: T \right)}{X} - \pdv{u\left( X,\: T \right)}{X}.
    \end{equation*}
    Here we have shown that this nonlinear PDE is invariant under the
    above dilation transformation when \(\beta = 1\) and \(\gamma = -1\).
\end{example}
\subsection{Translation Symmetry}
A translation symmetry is a transformation of the form
\begin{equation*}
    X = x - x_0, \quad T = t - t_0, \quad U = u - u_0,
\end{equation*}
where \(x_0\), \(t_0\), and \(u_0\) are constants.
\begin{example}
    Consider the heat equation
    \begin{equation*}
        \pdv{u\left( x,\: t \right)}{t} = D \pdv[order=2]{u\left( x,\: t \right)}{x}
    \end{equation*}
    and the mapping
    \begin{equation*}
        X = x - x_0, \quad T = t - t_0, \quad U = u - u_0.
    \end{equation*}
    By substituting the transformed solution into the PDE, we find
    \begin{align*}
        \pdv*{\left( u\left( X,\: T \right) - u_0 \right)}{t} & = D \pdv*[order=2]{\left( u\left( X,\: T \right) - u_0 \right)}{x}      \\
        \pdv{u\left( X,\: T \right)}{T} \odv{T}{t}            & = D \pdv*{\left( \pdv{u\left( X,\: T \right)}{X} \odv{X}{x} \right)}{x} \\
        \pdv{u\left( X,\: T \right)}{T}                       & = D \pdv{}{X} \odv{X}{x} \left( \pdv{u\left( X,\: T \right)}{X} \right) \\
        \pdv{u\left( X,\: T \right)}{T}                       & = D \pdv[order=2]{u\left( X,\: T \right)}{X}.
    \end{align*}
    Here we have shown that the heat equation is invariant under the
    above translation transformation.
\end{example}
\section{Similarity Solutions}
In the previous section, we considered dilation transformations where
both independent variables were scaled by some power of \(a\). Notice
that the product \(\eta = x t^{-1/\beta}\) is invariant under the same
dilation transformation:
\begin{align*}
    \eta = x t^{-1/\beta} \implies H = \left( \frac{x}{a} \right) \left( \frac{T}{a^\beta} \right)^{-1/\beta} = X T^{-1/\beta}.
\end{align*}
Let us therefore consider transformations of the form
\begin{equation*}
    X = a x, \quad T = a^\beta t, \quad U = t^{-\alpha} f\left( \eta \right),
\end{equation*}
where \(a\), \(\alpha\), and \(\beta\) are constant and \(f\) is an arbitrary
function to be determined. \(f\) is called a \textit{similarity solution}
(with \textit{similarity variable} \(\eta\)) and our goal is to
transform the PDE into an ODE with a single independent variable \(\eta\).
To do this:
\begin{enumerate}
    \item Substitute \(u\left( X,\: T \right)\) into the PDE to solve
          for \(\alpha\) and \(\beta\) while maintaining invariance.
    \item Obtain an ODE in terms of \(f\left( \eta \right)\) by
          substituting the similarity solution into the PDE.
    \item Obtain boundary conditions in terms of \(\eta\) using the
          same transformation.
    \item Solve the ODE to find \(f\left( \eta \right)\).
    \item Transform back to the original variables to find a solution
          to the PDE.
\end{enumerate}
\section{Travelling Wave Solutions}
Another type of solution is the \textit{travelling wave solution},
where the solution to a PDE appears to move at a constant velocity when
after a long period of time. This is similar to a steady state solution
where the solution does not change after a long period of time. In this
problem, we assume solutions of the form
\begin{equation*}
    u\left( x,\: t \right) = f\left( z \right), \quad z = x - c t,
\end{equation*}
where \(c\) is the speed at which the solution travels. The travelling
wave speed is often determined using analysis, boundary conditions,
numerical methods, or physical constraints. This method is only
applicable to PDEs that are invariant under translations to both
independent variables. As such, the PDE must not contain any explicit
dependence on \(x\) or \(t\).
\part{Method of Characteristics}
The method of characteristics is a technique used to solve nonlinear
PDEs by reducing them to a system of ODEs. It considers
parametrisations of the solution through a set of curves in the
solution space along which the solution is constant.
\section{First Order PDEs}
\subsection{Linearity}
Consider the general first order PDE in two variables:
\begin{equation*}
    a \pdv{u}{x} + b \pdv{u}{y} = c.
\end{equation*}
This PDE has four classifications based on the value of the
variables \(a\), \(b\), and \(c\):
\begin{itemize}
    \item \textbf{Linear}:
          \begin{equation*}
              a\left( x,\: y \right) \pdv{u}{x} + b\left( x,\: y \right) \pdv{u}{y} = c\left( x,\: y \right).
          \end{equation*}
    \item \textbf{Semi-Linear}:
          \begin{equation*}
              a\left( x,\: y \right) \pdv{u}{x} + b\left( x,\: y \right) \pdv{u}{y} = c\left( x,\: y,\: u \right).
          \end{equation*}
    \item \textbf{Quasi-Linear}:
          \begin{equation*}
              a\left( x,\: y,\: u \right) \pdv{u}{x} + b\left( x,\: y,\: u \right) \pdv{u}{y} = c\left( x,\: y,\: u \right).
          \end{equation*}
    \item \textbf{Nonlinear}: Otherwise.
\end{itemize}
\subsection{Solution Method}
Consider a quasi-linear PDE of the form:
\begin{equation*}
    a\left( x,\: y,\: u \right) \pdv{u}{x} + b\left( x,\: y,\: u \right) \pdv{u}{y} = c\left( x,\: y,\: u \right)
\end{equation*}
with initial data \(u_0\left( x \right)\) and initial condition
\(u\left( x_0,\: y_0 \right) = u_0\left( x \right)\), where \(a\), \(b\),
and \(c\) are continuous functions in \(x\), \(y\), and \(u\). To reduce
this problem into a family of ODEs, we will consider the solution
surface \(u = u\left( x,\: y \right)\) on which the PDE is satisfied.
If we express is surface implicitly as \(f\left( x,\: y,\: u \right) = u\left( x,\: y \right) - u = 0\),
we find the following normal vector:
\begin{equation*}
    \symbf{n} = \symbf{\nabla} f =
    \begin{bmatrix}
        u_x \\
        u_y \\
        -1
    \end{bmatrix}
    .
\end{equation*}
We will also define the vector field \(\symbf{v}\) consisting of the
coefficients of this PDE:
\begin{equation*}
    \symbf{v} =
    \begin{bmatrix}
        a\left( x,\: y,\: u \right) \\
        b\left( x,\: y,\: u \right) \\
        c\left( x,\: y,\: u \right)
    \end{bmatrix}
\end{equation*}
so that we express the PDE as \(\symbf{v} \cdot \symbf{n} = 0\). As
\(\symbf{n}\) is normal to the solution surface \(u\left( x,\: y \right)\),
\(\symbf{v}\) must lie on the tangent plane to the solution surface.
This vector field is therefore known as the \textit{characteristic
    direction} of the PDE. Let us now consider a parametric curve defined as:
\begin{equation*}
    \symbf{r}\left( s \right) =
    \begin{bmatrix}
        x\left( s \right) \\
        y\left( s \right) \\
        u\left( s \right)
    \end{bmatrix}
    ,
\end{equation*}
and let us impose that it's tangent vector is equal to \(\symbf{v}\)
so that the curve also lies on the solution surface:
\begin{equation*}
    \symbf{r}'\left( s \right) = \symbf{v}.
\end{equation*}
Doing so yields the following system of ODEs:
\begin{align*}
    \displaystyle\pdv{x}{s} & = a\left( x,\: y,\: u \right), \\
    \displaystyle\pdv{y}{s} & = b\left( x,\: y,\: u \right), \\
    \displaystyle\pdv{u}{s} & = c\left( x,\: y,\: u \right).
\end{align*}
We can solve these parametric curves through integration and solve for
all integration constants \(x_0\), \(y_0\), and \(u_0\) using the
parametric initial condition:
\begin{equation*}
    u\left( 0 \right) = u_0\left( \xi \right) \quad \text{on} \quad x\left( 0 \right) = x_0\left( \xi \right) \quad \text{and} \quad y\left( 0 \right) = y_0\left(\xi \right).
\end{equation*}
where \(\xi\) parametrises the initial condition. The resulting parametric
equations \(x\left( s \right)\) and \(y\left( s \right)\) form
\textit{characteristics} for the PDE, which allows us to determine the
value of \(u\left( x,\: y \right)\) along the curve for specific values
of \(\xi\). The solution \(u\left( x,\: y \right)\) can then be found by
eliminating \(s\) and \(\xi\) from the three parametric equations.
\subsection{Characteristic Curves}
After determining characteristics for a PDE, we can plot these
parametric curves in the \(x\)-\(y\) plane to determine regions where
the solution is constant. We do so by considering the \textit{domain of
dependence} for points in each region to determine the value of the
solution \(u\) by walking backward along these curves. Doing so allows
us to understand how the solution behaves along different
characteristics without needing to plot the entire solution surface.
\subsubsection{Expansion Waves}
Consider a quasi-linear PDE of the form
\begin{equation*}
    a\left( x,\: t,\: u \right) \pdv{u}{x} + \pdv{u}{t} = c\left( x,\: t,\: u \right)
\end{equation*}
with some initial condition. If this initial condition causes
characteristics to change slope when \(x_0 = 0\), so that they fan
outwards from this point, the region is called an \textit{expansion wave}.
For the solution to be defined at all points, we assume that
\(u\left( x_0,\: 0 \right)\) takes all values between \(u\left( x_0^-,\: 0 \right)\) and \(u\left( x_0^+,\: 0 \right)\),
on \(x\left( s \right)\). This means that as time increases, the
discontinuity in the initial condition spreads along the \(x\)-axis.
\subsubsection{Shock Waves}
Consider a quasi-linear PDE of the form
\begin{equation*}
    a\left( x,\: t,\: u \right) \pdv{u}{x} + \pdv{u}{t} = c\left( x,\: t,\: u \right)
\end{equation*}
with some initial condition. If this initial condition causes
characteristics to change slope at the point \(x_0 = 0\), where the
slopes of characteristics for \(x_0 < 0\) are steeper than those when
\(x_0 > 0\), we have an intersection of characteristics.
This results in a \textit{shock} in the solution, so that the solution
overtakes itself as time increases, leading to a multi-valued solution.
\section{System of First Order PDEs}
Consider an \(n \times n\) coupled system of first order PDEs:
\begin{equation*}
    \symbf{A} \pdv{\symbf{u}}{x} + \symbf{B} \pdv{\symbf{u}}{y} = \symbf{c}, \quad \text{with} \quad \symbf{u}\left( x_0,\: y_0 \right) = \symbf{u}_0\left( x \right),
\end{equation*}
where \(\symbf{A} = \symbf{A}\left( x,\: y,\: \symbf{u} \right)\) and
\(\symbf{B} = \symbf{B}\left( x,\: y,\: \symbf{u} \right)\) are
\(n \times n\) matrix functions, and
\(\symbf{c} = \symbf{c}\left( x,\: y,\: \symbf{u} \right)\) is an
\(n \times 1\) vector function.
In this case, we want to find characteristic directions \(\symbf{v}\)
that will allow us to decouple the system into first order ODEs. Let us
try to decompose \(\symbf{A}\) and \(\symbf{B}\) into a diagonal form by
assuming the following relationship holds for some \(\symbf{m}\):
\begin{equation*}
    \symbf{v}^\top \left( \symbf{A} \pdv{\symbf{u}}{x} + \symbf{B} \pdv{\symbf{u}}{y} \right) = \symbf{m}^\top \left( \alpha \pdv{\symbf{u}}{x} + \beta \pdv{\symbf{u}}{y} \right),
\end{equation*}
where \(\symbf{v}\) and \(\symbf{m}\) are \(n \times 1\) vector
functions, and \(\alpha = \odv{x}{s}\) and \(\beta = \odv{y}{s}\) are
scalar parametric functions. For this to hold, we must have:
\begin{equation*}
    \symbf{v}^\top \symbf{A} = \symbf{m}^\top \alpha \quad \text{and} \quad \symbf{v}^\top \symbf{B} = \symbf{m}^\top \beta.
\end{equation*}
By eliminating \(\symbf{m}^\top\), we find the following relationship
between \(\symbf{A}\) and \(\symbf{B}\):
\begin{align*}
    \frac{1}{\alpha} \symbf{v}^\top \symbf{A} & = \frac{1}{\beta} \symbf{v}^\top \symbf{B}      \\
    \symbf{v}^\top \symbf{A}                  & = \frac{\alpha}{\beta} \symbf{v}^\top \symbf{B} \\
    \symbf{v}^\top \symbf{A}                  & = \lambda \symbf{v}^\top \symbf{B}.
\end{align*}
This is precisely the left-generalised eigenvalue problem for the matrix
pair \(\left( \symbf{A},\: \symbf{B} \right)\) where the eigenvalues
\(\lambda_i\) are found by solving the characteristic equation:
\begin{equation*}
    \det{\left( \symbf{A} - \lambda \symbf{B} \right)} = 0.
\end{equation*}
Assuming a diagonalisable system, we have the following matrix
decomposition:
\begin{equation*}
    \symbf{A} = \symbf{B} \symbf{V} \symbf{\Lambda} \symbf{V}^{-1},
\end{equation*}
where \(\symbf{V}\) is the matrix of eigenvectors and \(\symbf{\Lambda}\)
is the diagonal matrix of eigenvalues. If we assume a solution of the
form \(\symbf{u} = \symbf{V} \symbf{w}\), we can transform the system
of PDEs into a diagonal form:
\begin{align*}
    \symbf{A} \pdv{\symbf{u}}{x} + \symbf{B} \pdv{\symbf{u}}{y}                                                                             & = \symbf{c}                                \\
    \left( \symbf{B} \symbf{V} \symbf{\Lambda} \symbf{V}^{-1} \right) \symbf{V} \pdv{\symbf{w}}{x} + \symbf{B} \symbf{V} \pdv{\symbf{w}}{y} & = \symbf{c}                                \\
    \symbf{B} \symbf{V} \symbf{\Lambda} \pdv{\symbf{w}}{x} + \symbf{B} \symbf{V} \pdv{\symbf{w}}{y}                                         & = \symbf{c}                                \\
    \symbf{\Lambda} \pdv{\symbf{w}}{x} + \pdv{\symbf{w}}{y}                                                                                 & = \symbf{V}^{-1} \symbf{B}^{-1} \symbf{c},
\end{align*}
where we assume \(\symbf{B}\) is invertible. Here the initial conditions
for \(\symbf{w}\) are determined by the initial conditions for
\(\symbf{u}\):
\begin{equation*}
    \symbf{w}\left( x_0,\: y_0 \right) = \symbf{V}^{-1} \symbf{u}_0\left( x \right).
\end{equation*}
As \(\symbf{\Lambda}\) is diagonal, we can form a system of \(n\) first
order PDEs which can be solved independently to find \(w_i\left( x,\: y \right)\),
allowing us to find the solution \(\symbf{u}\left( x,\: y \right)\).
\section{Second Order PDEs}
Consider the general linear second order PDE in two variables:
\begin{equation*}
    a \pdv[order=2]{u}{x} + 2b \pdv{u}{x,y} + c \pdv[order=2]{u}{y} + d \pdv{u}{x} + e \pdv{u}{y} = f.
\end{equation*}
with the initial conditions \(u\left( x_0,\: y_0 \right) = f\left( x \right)\)
and \(u_y\left( x_0,\: y_0 \right) = g\left( x \right)\) where variables
\(a\) through \(f\) are functions of \(x\) and \(y\) only. We will begin
by factoring this PDE into it's two first derivatives
\(u_x = \pdv{u}{x}\) and \(u_y = \pdv{u}{y}\) to convert the second
order PDE into a system of first order PDEs:
\begin{align*}
    a \pdv[order=2]{u}{x} + 2b \pdv{u}{y,x} + c \pdv[order=2]{u}{y} + d \pdv{u}{x} + e \pdv{u}{y} & = f                  \\
    a \pdv{u_x}{x} + 2b \pdv{u_x}{y} + c \pdv{u_y}{y} + d u_x + e u_y                             & = f                  \\
    a \pdv{u_x}{x} + 2b \pdv{u_x}{y} + c \pdv{u_y}{y}                                             & = f - d u_x - e u_y.
\end{align*}
Additionally, we will assume that Schwarz's theorem holds:
\begin{equation*}
    \pdv{u}{y,x} = \pdv{u}{x,y} \iff \pdv{u_x}{y} = \pdv{u_y}{x}.
\end{equation*}
This allows us to form the following system of first order PDEs:
\begin{equation*}
    \begin{bmatrix}
        a & 0 \\
        0 & 1
    \end{bmatrix}
    \pdv{}{x}
    \begin{bmatrix}
        u_x \\
        u_y
    \end{bmatrix}
    +
    \begin{bmatrix}
        2b & c \\
        -1 & 0
    \end{bmatrix}
    \pdv{}{y}
    \begin{bmatrix}
        u_x \\
        u_y
    \end{bmatrix}
    =
    \begin{bmatrix}
        f - d u_x - e u_y \\
        0
    \end{bmatrix}
    ,
\end{equation*}
which can be solved using the process described in the previous section,
where the initial condition for \(u_x\) is simply
\(u_x\left( x_0,\: y_0 \right) = f'\left( x \right)\).
\subsection{Classification of Second Order PDEs}
Using the above form, we can determine the characteristic equation for
a second order PDE:
\begin{equation*}
    \begin{vmatrix}
        a - 2b \lambda & -c       \\
        1              & -\lambda
    \end{vmatrix}
    = a \lambda^2 - 2b \lambda + c = 0,
\end{equation*}
where the eigenvalues \(\lambda\) are given by:
\begin{equation*}
    \lambda = \frac{2b \pm \sqrt{4b^2 - 4ac}}{2a} = \frac{b \pm \sqrt{b^2 - ac}}{a}.
\end{equation*}
We can classify this second order PDE into 3 types based on the sign of
the discriminant \(b^2 - ac\):
\begin{itemize}
    \item \textbf{Hyperbolic}: (two real distinct eigenvalues)
          \begin{equation*}
              b^2 - ac > 0.
          \end{equation*}
    \item \textbf{Parabolic}: (two real equal eigenvalues)
          \begin{equation*}
              b^2 - ac = 0.
          \end{equation*}
    \item \textbf{Elliptic}: (two complex eigenvalues)
          \begin{equation*}
              b^2 - ac < 0.
          \end{equation*}
\end{itemize}
\subsection{D'Alembert's Solution}
D'Alembert's solution is a general solution to the one-dimensional wave
equation
\begin{equation*}
    \pdv[order=2]{u}{t} = c^2 \pdv[order=2]{u}{x}, \quad -\infty < x < \infty, \quad t > 0,
\end{equation*}
with initial conditions \(u\left( x,\: 0 \right) = f\left( x \right)\)
and \(u_t\left( x,\: 0 \right) = g\left( x \right)\). Using the
techniques above, it can be shown that the general solution to this PDE
has the form:
\begin{equation*}
    u\left( x,\: t \right) = \frac{1}{2} \left[ f\left( x + c t \right) + f\left( x - c t \right) \right] + \frac{1}{2c} \int_{x - c t}^{x + c t} g\left( s \right) \odif{s}.
\end{equation*}
\section{Conservation of Mass}\label{sec:conservation-of-mass}
Consider the flow of some mass in a one-dimensional channel between \(x
= a\) and \(x = b\). Let \(v = v\left( x,\: t \right)\) be the velocity
of the flow, and \(\rho = \rho\left( x,\: t \right)\) be the density of
the flow, where velocity has units: length per time, and density has
units: mass over volume. Then, the mass flux \(q = q\left( x,\: t
\right)\) is defined as \(q = \rho v\), which has units: mass per unit
time per unit area.

The mass of the fluid per unit area between \(x = a\) and \(x = b\) is
given by the integral,
\begin{equation*}
    \int_a^b \rho \odif{x}.
\end{equation*}
Additionally, this mass is a dynamic quantity that depends on the flux
in at \(x = a\) given by:
\begin{equation*}
    \left. q \right|_{x = a}
\end{equation*}
and the mass flux out at \(x = b\) given by:
\begin{equation*}
    \left. q \right|_{x = b}.
\end{equation*}
The conservation of mass argument for this mass is a relationship
between these fluxes and the rate of change of mass in the channel:
\begin{align*}
    \odv*{\int_a^b \rho \odif{x}}{t}                & = \left. q \right|_{x = a} - \left. q \right|_{x = b} \\
    \int_a^b \pdv{\rho}{t} \odif{x}                 & = \int_b^a \pdv{q}{x} \odif{x}                        \\
    \int_a^b \pdv{\rho}{t} \odif{x}                 & = - \int_a^b \pdv{q}{x} \odif{x}                      \\
    \pdv{\rho}{t}                                   & = - \pdv{q}{x}                                        \\
    \pdv{\rho}{t} + \pdv{q}{x}                      & = 0                                                   \\
    \pdv{\rho}{t} + \pdv*{\left( \rho v \right)}{x} & = 0.
\end{align*}
We can also allow \(a\) and \(b\) to be functions of time, where we can
apply Leibniz integral rule to arrive at the same result on the interval
\(a\left( t \right) < x < b\left( t \right)\).
\subsection{Traffic Flow}
Consider a simple model for traffic flow where \(\rho\) is the density
of the flow of cars (number of cars per unit length), and \(q\) is the
flux of cars (number of cars per unit time), where \(q = \rho v\)
(\(v\) velocity). The number of cars between \(x = a\) and \(x = b\) is
given by:
\begin{equation*}
    \int_a^b \rho \odif{x},
\end{equation*}
the flux of cars in at \(x = a\) is given by \(\left. q \right|_{x = a}\),
and the flux of cars out at \(x = b\) is given by \(\left. q \right|_{x = b}\).
Then, for the conservation of the number of cars, we must have that:
\begin{equation*}
    \pdv{\rho}{t} + \pdv{q}{x} = 0.
\end{equation*}
This model is a PDE with two unknown variables, \(\rho\) and \(q\).
To close this system, we must introduce a constitutive relationship
between \(\rho\) and \(q\) (or between \(\rho\) and \(v\)), which is
determined through empirical analysis.
\subsection{Shock Solutions}
If a quasi-linear PDE inhibits a shock wave, and we desire a
single-valued solution instead, we can introduce a discontinuity in the
solution. To do so, we must assume that the total flow \(q = u v\) is
conserved across the \textit{shock wave} \(x_s\left( t \right)\), so
that the flow from the left into the moving shock wave equals the flow
to the right away from the shock wave.

Suppose conservation of mass holds between \(x = a\) and \(x = b\),
except at a shock wave at \(x = x_s\left( t \right)\), across which
both \(u\) and \(q\) are discontinuous. We can still expect a weak form
of conservation of mass to hold across the shock wave,
\begin{equation*}
    \odv*{\int_a^b u \odif{x}}{t} = \left. q \right|_{x = a} - \left. q \right|_{x = b}.
\end{equation*}
Consider dividing this integral into two regions that do not contain
the shock wave. Then, using Leibniz's rule, we find
\begin{align*}
    \odv*{\int_a^b u \odif{x}}{t} & = \odv*{\int_a^{x_s^{-}\left( t \right)} u \odif{x}}{t} + \odv*{\int_{x_s^{+}\left( t \right)}^b u \odif{x}}{t}                                                                                       \\
                                  & = \int_a^{x_s^{-}\left( t \right)} \pdv{u}{t} \odif{x} + \left. u \right|_{x = x^{-}} \odv{x_s}{t} + \int_{x_s^{+}\left( t \right)}^b \pdv{u}{t} \odif{x} - \left. u \right|_{x = x^{+}} \odv{x_s}{t}
\end{align*}
For the RHS, we can use the fundamental theorem of calculus to find:
\begin{align*}
    \left. q \right|_{x = a} - \left. q \right|_{x = b} & = \left. q \right|_{x = a} - \left. q \right|_{x = x^{-}} + \left. q \right|_{x = x^{+}} - \left. q \right|_{x = b} + \left. q \right|_{x = x^{-}} - \left. q \right|_{x = x^{+}} \\
                                                        & = \left. q \right|_{x = x^{-}}^{x = a} + \left. q \right|_{x = b}^{x = x^{+}} + \left. q \right|_{x = x^{-}} - \left. q \right|_{x = x^{+}}                                       \\
                                                        & = \int_{x_s^{-}\left( t \right)}^a \pdv{q}{x} \odif{x} + \int_b^{x_s^{+}\left( t \right)} \pdv{q}{x} \odif{x} + \left. q \right|_{x = x^{-}} - \left. q \right|_{x = x^{+}}       \\
                                                        & = -\int_a^{x_s^{-}\left( t \right)} \pdv{q}{x} \odif{x} - \int_{x_s^{+}\left( t \right)}^b \pdv{q}{x} \odif{x} + \left. q \right|_{x = x^{-}} - \left. q \right|_{x = x^{+}}.
\end{align*}
By equating the LHS and RHS, we find
\begin{align*}
    \int_a^{x_s^{-}\left( t \right)} \pdv{u}{t} + \pdv{q}{x} \odif{x} + \int_{x_s^{+}\left( t \right)}^b \pdv{u}{t} + \pdv{q}{x} \odif{x} + \left. u \right|_{x = x^{-}} \odv{x_s}{t} - \left. u \right|_{x = x^{+}} \odv{x_s}{t} & = \left. q \right|_{x = x^{-}} - \left. q \right|_{x = x^{+}}                                                                      \\
    \odv{x_s}{t} \left[ \left. u \right|_{x = x^{-}} - \left. u \right|_{x = x^{+}} \right]                                                                                                                                       & = \left. q \right|_{x = x^{-}} - \left. q \right|_{x = x^{+}}                                                                      \\
    \odv{x_s}{t}                                                                                                                                                                                                                  & = \frac{\left. q \right|_{x = x^{-}} - \left. q \right|_{x = x^{+}}}{\left. u \right|_{x = x^{-}} - \left. u \right|_{x = x^{+}}}.
\end{align*}
Or, more compactly,
\begin{equation*}
    x_s'\left( t \right) = \frac{q^+ - q^-}{u^+ - u^-} = \frac{\left[ q \right]}{\left[ u \right]} = \frac{\text{jump in flow}}{\text{jump in value}}.
\end{equation*}
By solving for the shock wave \(x_s\left( t \right)\), we can divide
the solution into two regions, one on either side of the shock wave.
Characteristics will then meet at this shock wave leading to a
single-valued solution. Note this solution does not have a derivative at
the shock wave, and the shock solution is not unique.
\subsection{Caustics}
If the characteristics of a PDE intersect along some curve not arising
from the \(x\)-axis, we have a \textit{caustic}. Such characteristics
may lead to more than 2 values for the solution. When determining a
shock solution for such cases, we must consider the time \(t_s\) at
which the solution first becomes multi-valued. \(t_s\) is the time at
which the slope:
\begin{equation*}
    \pdv{u}{x} = \frac{\pdv{u}{\xi}}{\pdv{x\left( s \right)}{\xi}}
\end{equation*}
becomes infinite. This can be found by solving for the value of \(s\) at
which the denominator of the above expression becomes zero. If we call
this value \(s_{\text{caustic}}\), we can parametrically define the
caustic as:
\begin{equation*}
    \symbf{x}_{\text{caustic}}\left( \xi \right) =
    \begin{bmatrix}
        x\left( s_{\text{caustic}} \right) \\
        t\left( s_{\text{caustic}} \right)
    \end{bmatrix}
    =
    \begin{bmatrix}
        x_{\text{caustic}} \left( \xi \right) \\
        t_{\text{caustic}} \left( \xi \right)
    \end{bmatrix}
\end{equation*}
We can then use the tangent vector \(\symbf{x}_{\text{caustic}}'\left( \xi \right)\)
to determine the minimum of this curve to find the first time \(t_s\) at
which the solution becomes multi-valued. When solving for the shock
\(x_s\left( t \right)\) as outlined in the previous section, we can use
this initial value to find the constant of integration.
\subsection{Higher-Order Terms}
If we wish for a shock to be resolved without a jump discontinuity, we
can introduce higher-order terms into the PDE that will smooth out the
solution.
\part{Viscous Flow}
In the following sections, we will discuss the viscous flow of
incompressible Newtonian fluids. We will begin by deriving the governing
equations for fluid flow using methods from continuum mechanics. Here we
treat fluids as continuous media, where fluid properties are described
by fields that vary continuously in space and time.
\section{Continuum Mechanics}
\subsection{Material Derivative}
Consider the temporal rate of change of a quantity which follows the
motion of a fluid particle at position \(\symbf{r}\left( x,\: y,\: z,\:
t \right)\). We cannot simply take the partial derivative of the field
with respect to time as it is not fixed in space. Instead, we will
introduce the \textit{material derivative}, which accounts for the
change in a field as a fluid particle moves through space. We will
motivate this definition by deriving the acceleration of a fluid
particle. Using the limit definition of the derivative, we find:
\begin{equation*}
    \symbf{a}\left( x,\: y,\: z,\: t \right) = \lim_{\adif{t} \to 0} \frac{\symbf{q}\left( x + \adif{x},\: y + \adif{y},\: z + \adif{z},\: t + \adif{t} \right) - \symbf{q}\left( x,\: y,\: z,\: t \right)}{\Delta t},
\end{equation*}
where \(\symbf{q}\) is the velocity field of the fluid, defined:
\begin{equation*}
    \symbf{q}\left( x,\: y,\: z,\: t \right) =
    \begin{bmatrix}
        u\left( x,\: y,\: z,\: t \right) \\
        v\left( x,\: y,\: z,\: t \right) \\
        w\left( x,\: y,\: z,\: t \right)
    \end{bmatrix}
    =
    \odv{\symbf{r}}{t}.
\end{equation*}
Notice that a small step in time has also resulted in a small step in
space. Let us therefore use the Taylor series expansion of the velocity
field to simplify the numerator of the above expression to:
\begin{equation*}
    \symbf{q}\left( x + \adif{x},\: y + \adif{y},\: z + \adif{z},\: t + \adif{t} \right) = \symbf{q}\left( x,\: y,\: z,\: t \right) + \pdv{\symbf{q}}{x} \adif{x} + \pdv{\symbf{q}}{y} \adif{y} + \pdv{\symbf{q}}{z} \adif{z} + \pdv{\symbf{q}}{t} \adif{t} + \ldots,
\end{equation*}
where we are evaluating the velocity field at the point \(\left( x,\:
y,\: z,\: t \right)\). Substituting this into the numerator, we find:
\begin{align*}
    \symbf{a}\left( x,\: y,\: z,\: t \right) & = \lim_{\adif{t} \to 0} \left[ \pdv{\symbf{q}}{x} \adv{x}{t} + \pdv{\symbf{q}}{y} \adv{y}{t} + \pdv{\symbf{q}}{z} \adv{z}{t} + \pdv{\symbf{q}}{t} + \ldots \right] \\
                                             & = \pdv{\symbf{q}}{x} \odv{x}{t} + \pdv{\symbf{q}}{y} \odv{y}{t} + \pdv{\symbf{q}}{z} \odv{z}{t} + \pdv{\symbf{q}}{t}                                               \\
                                             & = \pdv{\symbf{q}}{x} u + \pdv{\symbf{q}}{y} v + \pdv{\symbf{q}}{z} w + \pdv{\symbf{q}}{t}                                                                          \\
                                             & = \pdv{\symbf{q}}{t} + u \pdv{\symbf{q}}{x} + v \pdv{\symbf{q}}{y} + w \pdv{\symbf{q}}{z}                                                                          \\
                                             & = \pdv{\symbf{q}}{t} + \left( u \pdv{}{x} + v \pdv{}{y} + w \pdv{}{z} \right) \symbf{q}                                                                            \\
                                             & = \pdv{\symbf{q}}{t} + \left( \symbf{q} \cdot \symbf{\nabla} \right) \symbf{q}.
\end{align*}
\begin{definition}[Material Derivative]
    The rate of change of a quantity as observed from a moving reference
    frame following a fluid particle can be expressed as the material
    derivative of the quantity. This operator is denoted as:
    \begin{equation*}
        \mdv{}{t} = \pdv{}{t} + \symbf{q} \cdot \symbf{\nabla}.
    \end{equation*}
\end{definition}
Therefore, we can express the acceleration of a fluid particle in a
moving reference frame as:
\begin{equation*}
    \symbf{a} = \mdv{\symbf{q}}{t}.
\end{equation*}
\subsection{Conservation of Mass}
Let us now consider what a conservation of mass argument would look
like for a fluid. For a control volume \(V\), the total mass of the
fluid is given by the integral:
\begin{equation*}
    M = \iiint_V \rho \odif{V},
\end{equation*}
where \(\rho = \rho\left( x,\: y,\: z,\: t \right)\) is the density of
the fluid. As we saw in the previous
section, we want the rate of change of mass to be equal to the net flux
of fluid into the control volume. This can be expressed using a surface
integral over the surface of the control volume:
\begin{equation*}
    \odv{M}{t} = -\oiint_{\partial V} \rho \left( \symbf{q} \cdot \symbf{n} \right) \odif{\symbf{\sigma}},
\end{equation*}
where \(\symbf{n}\) is the outward unit normal vector to the surface of
the control volume, and \(\odif{\symbf{\sigma}}\) is the differential
area element. By applying the divergence theorem, we can rewrite this
surface integral as:
\begin{equation*}
    \odv*{\iiint_V \rho \odif{V}}{t} = -\iiint_V \symbf{\nabla} \cdot \left( \rho \symbf{q} \right) \odif{V}.
\end{equation*}
As the control volume is arbitrary and fixed, we can apply Leibniz's
integral rule to find:
\begin{align*}
    \iiint_V \pdv{\rho}{t} \odif{V} + \iiint_V \symbf{\nabla} \cdot \left( \rho \symbf{q} \right) \odif{V} & = 0  \\
    \iiint_V \left( \pdv{\rho}{t} + \symbf{\nabla} \cdot \left( \rho \symbf{q} \right) \right) \odif{V}    & = 0  \\
    \pdv{\rho}{t} + \symbf{\nabla} \cdot \left( \rho \symbf{q} \right)                                     & = 0.
\end{align*}
We can alternatively express this in terms of the material derivative,
by expanding the divergence term:
\begin{align*}
    \pdv{\rho}{t} + \symbf{\nabla} \cdot \left( \rho \symbf{q} \right)                                                        & = 0  \\
    \pdv{\rho}{t} + \odv{\rho u}{x} + \odv{\rho v}{y} + \odv{\rho w}{z}                                                       & = 0  \\
    \pdv{\rho}{t} + \odv{\rho}{x} u + \rho \odv{u}{x} + \odv{\rho}{y} v + \rho \odv{v}{y} + \odv{\rho}{z} w + \rho \odv{w}{z} & = 0  \\
    \pdv{\rho}{t} + \odv{\rho}{x} u + \odv{\rho}{y} v + \odv{\rho}{z} w + \rho \odv{u}{x} + \rho \odv{v}{y} + \rho \odv{w}{z} & = 0  \\
    \pdv{\rho}{t} + \symbf{\nabla} \rho \cdot \symbf{q} + \rho \symbf{\nabla} \cdot \symbf{q}                                 & = 0  \\
    \mdv{\rho}{t} + \rho \symbf{\nabla} \cdot \symbf{q}                                                                       & = 0.
\end{align*}
This result is known as the \textit{continuity equation} and expresses
the conservation of mass for a fluid.
\subsection{Conservation of Momentum}
Let us now consider the conservation of momentum for a fluid. Using
Newton's second law of motion, we can express the rate of change in
momentum as the sum of the forces acting on the fluid. For a control
volume \(V\), the rate of change in momentum of a fluid is given by
the integral:
\begin{equation*}
    \odv*{\iiint_V \rho \symbf{q} \odif{V}}{t} = \symbf{F}.
\end{equation*}
Here, \(\symbf{F}\) is the total force acting on the fluid, per unit
volume. In a fluid, we can decompose this force into the sum of
\textbf{body forces} \(\symbf{F}_b\), and \textbf{surface forces}
\(\symbf{F}_s\), that act on some differential fluid element \(\odif{V}\).
Body forces are uniformally distributed through an element and can
include gravitational forces, electromagnetic forces, coriolis forces,
etc. Here we will denote this generally as:
\begin{equation*}
    \odif{\symbf{F}_b} = \rho \symbf{g} \odif{V} \implies \symbf{F}_b = \iiint_V \rho \symbf{g} \odif{V},
\end{equation*}
where \(\symbf{g}\) is the acceleration due to some external body force.
Note that we do not multiply this by the mass of the fluid element, as
we are considering the force per unit volume. Surface forces act on the
surface of the fluid element and include pressure and viscous forces.
Here we will introduce the \textbf{total stress tensor} \(\symbf{\sigma}\)
which is the sum of \textbf{viscous stresses} \(\symbf{\tau}\) that
act both tangentially and normally to the surface of the fluid element,
and hydrostatic \textbf{pressures} \(p\) that act normally to the
surface of the fluid element. We can express this in component form as:
\begin{equation*}
    \sigma_{ij} = -p \delta_{ij} + \tau_{ij},
\end{equation*}
where \(\delta_{ij}\) is the Kronecker delta, or, using matrix notation:
\begin{align*}
    \begin{bmatrix}
        \sigma_{xx} & \sigma_{yx} & \sigma_{zx} \\
        \sigma_{xy} & \sigma_{yy} & \sigma_{zy} \\
        \sigma_{xz} & \sigma_{yz} & \sigma_{zz}
    \end{bmatrix}
                   & =
    \begin{bmatrix}
        -p + \tau_{xx} & \tau_{yx}      & \tau_{zx}      \\
        \tau_{xy}      & -p + \tau_{yy} & \tau_{zy}      \\
        \tau_{xz}      & \tau_{yz}      & -p + \tau_{zz}
    \end{bmatrix}
    \\
    \symbf{\sigma} & = -p \symbf{I} + \symbf{\tau}.
\end{align*}
The subscript \(\sigma_{ij}\) denotes a stress in the \(j\) direction
acting on a surface element with normal in the \(i\) direction. To find
the total surface force acting on a fluid element, consider all the
forces acting in the \(x\)-direction:
\begin{equation*}
    \odif{\symbf{F}_{sx}} = \left( - \sigma_{xx} + \sigma_{x+\adif{x},x} \right) \adif{y} \adif{z} + \left( - \sigma_{yx} + \sigma_{y+\adif{y},x} \right) \adif{x} \adif{z} + \left( - \sigma_{zx} + \sigma_{z+\adif{z},x} \right) \adif{x} \adif{y},
\end{equation*}
If we use the first-order Taylor expansion on these terms, this
simplifies to:
\begin{align*}
    \odif{\symbf{F}_{sx}} & = \pdv{\sigma_{xx}}{x} \adif{x} \adif{y} \adif{z} + \pdv{\sigma_{yx}}{y} \adif{y} \adif{x} \adif{z} + \pdv{\sigma_{zx}}{z} \adif{z} \adif{x} \adif{y} \\
                          & = \left( \pdv{\sigma_{xx}}{x} + \pdv{\sigma_{yx}}{y} + \pdv{\sigma_{zx}}{z} \right) \adif{V}.
\end{align*}
Therefore, for all three directions, we find:
\begin{align*}
    \odif{\symbf{F}_{sx}} & = \left( \pdv{\sigma_{xx}}{x} + \pdv{\sigma_{yx}}{y} + \pdv{\sigma_{zx}}{z} \right) \adif{V}, \\
    \odif{\symbf{F}_{sy}} & = \left( \pdv{\sigma_{xy}}{x} + \pdv{\sigma_{yy}}{y} + \pdv{\sigma_{zy}}{z} \right) \adif{V}, \\
    \odif{\symbf{F}_{sz}} & = \left( \pdv{\sigma_{xz}}{x} + \pdv{\sigma_{yz}}{y} + \pdv{\sigma_{zz}}{z} \right) \adif{V},
\end{align*}
written compactly as:
\begin{equation*}
    \odif{\symbf{F}_{si}} = \symbf{\nabla} \cdot \symbf{\sigma}_i^\top \odif{V}, \quad \text{where } \symbf{\sigma}_i = -p \symbf{e}_i^\top + \symbf{\tau}_i, \quad \text{for } i = x,\: y,\: z.
\end{equation*}
Here \(\symbf{\sigma}_i\) is the \(i\)-th row of the total stress tensor,
\(\symbf{e}_i\) is the \(i\)-th unit vector, and \(\symbf{\tau}_i\) is
the \(i\)-th row of the viscous stress tensor.
For Newtonian fluids, the viscous stress tensor can be related to the
velocity field of a fluid, by the \textit{Newtonian law of viscosity},
which states that the viscous stress is proportional to the rate of
strain of the fluid. This can be expressed as:
\begin{equation*}
    \tau_{ij} = \mu \left( \pdv{q_i}{x_j} + \pdv{q_j}{x_i} \right) \implies \symbf{\tau}_i = \mu
    \begin{bmatrix}
        \displaystyle \pdv{q_i}{x} + \pdv{u}{x_i} &
        \displaystyle \pdv{q_i}{y} + \pdv{v}{x_i} &
        \displaystyle \pdv{q_i}{z} + \pdv{w}{x_i}
    \end{bmatrix},
\end{equation*}
where \(\mu\) is the dynamic viscosity of the fluid and \(q_i\) is the
\(i\)-th component of the velocity field. This allows us to express the
total surface force acting on the fluid element in the \(i\)-th
direction as:
\begin{align*}
    \odif{\symbf{F}_{si}} & = \symbf{\nabla} \cdot \left( -p \symbf{e}_i + \symbf{\tau}_i^\top \right) \odif{V}                                                  \\
                          & = \left( - \pdv{p}{x_i} + \symbf{\nabla} \cdot \symbf{\tau}_i^\top \right) \odif{V}                                                  \\
                          & = \left( - \pdv{p}{x_i} + \mu \symbf{\nabla} \cdot
    \begin{bmatrix}
        \displaystyle \pdv{q_i}{x} + \pdv{u}{x_i} \\
        \displaystyle \pdv{q_i}{y} + \pdv{v}{x_i} \\
        \displaystyle \pdv{q_i}{z} + \pdv{w}{x_i}
    \end{bmatrix}
    \right) \odif{V}                                                                                                                                             \\
                          & = \left( - \pdv{p}{x_i} + \mu \left(
        \pdv[order=2]{q_i}{x} + \pdv{u}{x,x_i} +
        \pdv[order=2]{q_i}{y} + \pdv{v}{y,x_i} +
        \pdv[order=2]{q_i}{z} + \pdv{w}{z,x_i}
    \right) \right) \odif{V}                                                                                                                                     \\
                          & = \left( - \pdv{p}{x_i} + \mu \left(
        \pdv[order=2]{q_i}{x} + \pdv[order=2]{q_i}{y} + \pdv[order=2]{q_i}{z} +
        \pdv{u}{x,x_i} + \pdv{v}{y,x_i} + \pdv{w}{z,x_i}
    \right) \right) \odif{V}                                                                                                                                     \\
                          & = \left( - \pdv{p}{x_i} + \mu \symbf{\nabla}^2 q_i + \mu \symbf{\nabla} \cdot \pdv{\symbf{q}}{x_i} \right) \odif{V}                  \\
                          & = \left( - \pdv{p}{x_i} + \mu \symbf{\nabla}^2 q_i + \mu \pdv*{\left( \symbf{\nabla} \cdot \symbf{q} \right)}{x_i} \right) \odif{V}.
\end{align*}
By combining all three directions, we can write this using the gradient
operator:
\begin{equation*}
    \odif{\symbf{F}_s} = \left( -\symbf{\nabla} p + \mu \symbf{\nabla}^2 \symbf{q} + \mu \symbf{\nabla} \left( \symbf{\nabla} \cdot \symbf{q} \right) \right) \odif{V}.
\end{equation*}
By integrating this over the control volume, we find that the total
surface force acting on the fluid element is:
\begin{equation*}
    \symbf{F}_s = \iiint_V \left( -\symbf{\nabla} p + \mu \symbf{\nabla}^2 \symbf{q} + \mu \symbf{\nabla} \left( \symbf{\nabla} \cdot \symbf{q} \right) \right) \odif{V}.
\end{equation*}
Therefore, the conservation of momentum in a fluid can be expressed as:
\begin{align*}
    \odv*{\iiint_V \rho \symbf{q} \odif{V}}{t}        & = \symbf{F}_b + \symbf{F}_s                                                                                                                                             \\
    \iiint_V \mdv{\rho \symbf{q}}{t} \odif{V}         & = \iiint_V \left( \rho \symbf{g} -\symbf{\nabla} p + \mu \symbf{\nabla}^2 \symbf{q} + \mu \symbf{\nabla} \left( \symbf{\nabla} \cdot \symbf{q} \right) \right) \odif{V} \\
    \mdv{\rho}{t} \symbf{q} + \rho \mdv{\symbf{q}}{t} & = \rho \symbf{g} - \symbf{\nabla} p + \mu \symbf{\nabla}^2 \symbf{q} + \mu \symbf{\nabla} \left( \symbf{\nabla} \cdot \symbf{q} \right)                                 \\
\end{align*}
We can simplify this relationship further by assuming that the density
of the fluid is constant so that \(\mdv{\rho}{t} = 0\). Therefore, given
the following constitutive relationship for an incompressible fluid:
\begin{equation*}
    \symbf{\nabla} \cdot \symbf{q} = 0,
\end{equation*}
we find the governing equations for the flow of an incompressible
Newtonian fluid to be:
\begin{equation*}
    \rho \mdv{\symbf{q}}{t} = \rho \symbf{g} - \symbf{\nabla} p + \mu \symbf{\nabla}^2 \symbf{q}.
\end{equation*}
This result is known as the \textit{Navier-Stokes equations} for an
incompressible fluid
\end{document}
