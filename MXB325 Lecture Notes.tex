%!TEX TS-program = xelatex
%!TEX options = -aux-directory=Debug -shell-escape -file-line-error -interaction=nonstopmode -halt-on-error -synctex=1 "%DOC%"
\documentclass{article}
\input{LaTeX-Submodule/template.tex}

% Additional packages & macros
\theoremstyle{definition}
\newtheorem{example}{Example}[section]

% Header and footer
\newcommand{\unitName}{Differential Equations and Modelling 2}
\newcommand{\unitTime}{Semester 2, 2024}
\newcommand{\unitCoordinator}{Professor Scott McCue}
\newcommand{\documentAuthors}{Tarang Janawalkar}

\fancyhead[L]{\unitName}
\fancyhead[R]{\leftmark}
\fancyfoot[C]{\thepage}

% Copyright
\usepackage[
    type={CC},
    modifier={by-nc-sa},
    version={4.0},
    imagewidth={5em},
    hyphenation={raggedright}
]{doclicense}

\date{}

\begin{document}
%
\begin{titlepage}
    \vspace*{\fill}
    \begin{center}
        \LARGE{\textbf{\unitName}} \\[0.1in]
        \normalsize{\unitTime} \\[0.2in]
        \normalsize\textit{\unitCoordinator} \\[0.2in]
        \documentAuthors
    \end{center}
    \vspace*{\fill}
    \doclicenseThis
    \thispagestyle{empty}
\end{titlepage}
\newpage
%
\tableofcontents
\newpage
%
\part{Symmetry Methods}
\section{Symmetry Transformations}
Consider a partial differential equation for \(u\left( x,\: t \right)\)
whose domain lies in \(\R^2\). Such a problem typically does not have
any natural length or time scales associated with it's fundamental
solution. Thus, let us transform the independent and dependent
variables through the mapping \(\left( x,\: t,\: u \right) \mapsto
\left( X,\: T,\: U \right)\) where \(X = X\left( x,\: t,\: u \right)\),
\(T = T\left( x,\: t,\: u \right)\), and \(U = U\left( x,\: t,\: u
\right)\), such that the PDE is invariant under this transformation,
that is, the transformed PDE in \(U\) has the same form as the original
PDE in \(u\).

The choice of mapping may consist of dilations and translations of the
independent and dependent variables by some constant factors, where
each constant is enforced by the invariance condition. To do this, we
must ensure that the function \(U\left( x,\: t,\: u\left( X,\: T
\right) \right)\) satisfies the transformed PDE. When this is the case,
the PDE is said to have a \textit{symmetry transformation}.

The goal of these symmetry methods is to find an appropriate
transformation that simplies the PDE, often reducing it to an ODE.
\subsection{Dilation Symmetry}
A dilation symmetry is a transformation of the form
\begin{equation*}
    X = a x, \quad T = a^\beta t, \quad U = a^\gamma u,
\end{equation*}
where \(a\), \(\beta\), and \(\gamma\) are constants. Note that we do
not express the transformation for \(x\) as \(X = a^\alpha x\) since it
does not provide any additional information.
\begin{example}
    Consider the transport equation
    \begin{equation*}
        \pdv{u\left( x,\: t \right)}{t} + c \pdv{u\left( x,\: t \right)}{x} = 0
    \end{equation*}
    and the mapping
    \begin{equation*}
        X = a x, \quad T = a^\beta t, \quad U = u.
    \end{equation*}
    After substituting the transformed solution into the PDE, we find
    \begin{align*}
        \pdv{u\left( X,\: T \right)}{t} + c \pdv{u\left( X,\: T \right)}{x}                       & = 0  \\
        \pdv{u\left( X,\: T \right)}{T} \odv{T}{t} + c \pdv{u\left( X,\: T \right)}{X} \odv{X}{x} & = 0  \\
        a^\beta \pdv{u\left( X,\: T \right)}{T} + a c \pdv{u\left( X,\: T \right)}{X}             & = 0.
    \end{align*}
    For this PDE to be invariant under a dilation transformation, we
    must factor out the constant \(a\) from the equation. This implies
    that:
    \begin{equation*}
        a^\beta = a \implies \beta = 1,
    \end{equation*}
    so that
    \begin{equation*}
        \pdv{u\left( X,\: T \right)}{T} + c \pdv{u\left( X,\: T \right)}{X} = 0.
    \end{equation*}
    Here we have shown that the transport equation is invariant under
    the above dilation transformation when \(\beta = 1\).
\end{example}
\begin{example}
    Consider the nonlinear PDE
    \begin{equation*}
        \pdv{u\left( x,\: y \right)}{x} + y^2 \pdv{u\left( x,\: y \right)}{y} = 0
    \end{equation*}
    and the mapping
    \begin{equation*}
        X = a x, \quad Y = a^\beta y, \quad U = u.
    \end{equation*}
    By substituting the transformed solution into the PDE, we find
    \begin{align*}
        \pdv{u\left( X,\: Y \right)}{x} + y^2 \pdv{u\left( X,\: Y \right)}{y}                                    & = 0  \\
        \pdv{u\left( X,\: Y \right)}{X} \odv{X}{x} + y^2 \pdv{u\left( X,\: Y \right)}{Y} \odv{Y}{y}              & = 0  \\
        a \pdv{u\left( X,\: Y \right)}{X} + a^\beta y^2 \pdv{u\left( X,\: Y \right)}{Y}                          & = 0  \\
        a \pdv{u\left( X,\: Y \right)}{X} + a^\beta \left( Y / a^\beta \right)^2 \pdv{u\left( X,\: Y \right)}{Y} & = 0  \\
        a \pdv{u\left( X,\: Y \right)}{X} + a^{-\beta} Y^2 \pdv{u\left( X,\: Y \right)}{Y}                       & = 0.
    \end{align*}
    For this PDE to be invariant under a dilation transformation, we
    must again factor out \(a\) from the equation so that:
    \begin{equation*}
        a = a^{-\beta} \implies 1 = -\beta \implies \beta = -1,
    \end{equation*}
    and
    \begin{equation*}
        \pdv{u\left( X,\: Y \right)}{X} + Y^2 \pdv{u\left( X,\: Y \right)}{Y} = 0.
    \end{equation*}
    Here we have shown that this nonlinear PDE is invariant under the
    above dilation transformation when \(\beta = -1\).
\end{example}
\begin{example}
    Consider the nonlinear convection diffusion equation
    \begin{equation*}
        \pdv{u\left( x,\: t \right)}{t} = u\left( x,\: t \right) \pdv[order=2]{u\left( x,\: t \right)}{x} - \pdv{u\left( x,\: t \right)}{x}
    \end{equation*}
    and the mapping
    \begin{equation*}
        X = a x, \quad T = a^\beta t, \quad U = a^\gamma u.
    \end{equation*}
    We will once again substitute the transformed solution into the PDE
    to find
    \begin{align*}
        a^\gamma \pdv{u\left( X,\: T \right)}{t}            & = a^{2\gamma} u\left( X,\: T \right) \pdv[order=2]{u\left( X,\: T \right)}{x} - a^\gamma \pdv{u\left( X,\: T \right)}{x}                                       \\
        a^\gamma \pdv{u\left( X,\: T \right)}{T} \odv{T}{t} & = a^{2\gamma} u\left( X,\: T \right) \pdv*{\left[ \pdv{u\left( X,\: T \right)}{X} \odv{X}{x} \right]}{x} - a^\gamma \pdv{u\left( X,\: T \right)}{X} \odv{X}{x} \\
        a^{\beta + \gamma} \pdv{u\left( X,\: T \right)}{T}  & = a^{1 + 2\gamma} u\left( X,\: T \right) \pdv*{\left[ \odv{X}{x} \pdv{u\left( X,\: T \right)}{X} \right]}{X} - a^{1 + \gamma} \pdv{u\left( X,\: T \right)}{X}  \\
        a^{\beta + \gamma} \pdv{u\left( X,\: T \right)}{T}  & = a^{2 + 2\gamma} u\left( X,\: T \right) \pdv[order=2]{u\left( X,\: T \right)}{X} - a^{1 + \gamma} \pdv{u\left( X,\: T \right)}{X}.
    \end{align*}
    For this PDE to be invariant under a dilation transformation, we
    must factor out \(a\) from the equation so that:
    \begin{equation*}
        a^{\beta + \gamma} = a^{2 + 2\gamma} = a^{1 + \gamma} \implies \beta + \gamma = 2 + 2\gamma = 1 + \gamma \implies \beta = 1, \gamma = -1,
    \end{equation*}
    and
    \begin{equation*}
        \pdv{u\left( X,\: T \right)}{T} = u\left( X,\: T \right) \pdv[order=2]{u\left( X,\: T \right)}{X} - \pdv{u\left( X,\: T \right)}{X}.
    \end{equation*}
    Here we have shown that this nonlinear PDE is invariant under the
    above dilation transformation when \(\beta = 1\) and \(\gamma = -1\).
\end{example}
\subsection{Translation Symmetry}
A translation symmetry is a transformation of the form
\begin{equation*}
    X = x - x_0, \quad T = t - t_0, \quad U = u - u_0,
\end{equation*}
where \(x_0\), \(t_0\), and \(u_0\) are constants.
\begin{example}
    Consider the heat equation
    \begin{equation*}
        \pdv{u\left( x,\: t \right)}{t} = D \pdv[order=2]{u\left( x,\: t \right)}{x}
    \end{equation*}
    and the mapping
    \begin{equation*}
        X = x - x_0, \quad T = t - t_0, \quad U = u - u_0.
    \end{equation*}
    By substituting the transformed solution into the PDE, we find
    \begin{align*}
        \pdv*{\left( u\left( X,\: T \right) - u_0 \right)}{t} & = D \pdv*[order=2]{\left( u\left( X,\: T \right) - u_0 \right)}{x}      \\
        \pdv{u\left( X,\: T \right)}{T} \odv{T}{t}            & = D \pdv*{\left( \pdv{u\left( X,\: T \right)}{X} \odv{X}{x} \right)}{x} \\
        \pdv{u\left( X,\: T \right)}{T}                       & = D \pdv{}{X} \odv{X}{x} \left( \pdv{u\left( X,\: T \right)}{X} \right) \\
        \pdv{u\left( X,\: T \right)}{T}                       & = D \pdv[order=2]{u\left( X,\: T \right)}{X}.
    \end{align*}
    Here we have shown that the heat equation is invariant under the
    above translation transformation.
\end{example}
\section{Similarity Solutions}
In the previous section, we considered dilation transformations where
both independent variables were scaled by some power of \(a\). Notice
that the product \(\eta = x t^{-1/\beta}\) is invariant under the same
dilation transformation:
\begin{align*}
    \eta = x t^{-1/\beta} \implies H = \left( \frac{x}{a} \right) \left( \frac{T}{a^\beta} \right)^{-1/\beta} = X T^{-1/\beta}.
\end{align*}
Let us therefore consider transformations of the form
\begin{equation*}
    X = a x, \quad T = a^\beta t, \quad U = t^{-\alpha} f\left( \eta \right),
\end{equation*}
where \(a\), \(\alpha\), and \(\beta\) are constant and \(f\) is an arbitrary
function to be determined. \(f\) is called a \textit{similarity solution}
(with \textit{similarity variable} \(\eta\)) and our goal is to
transform the PDE into an ODE with a single independent variable \(\eta\).
To do this:
\begin{enumerate}
    \item Substitute \(u\left( X,\: T \right)\) into the PDE to solve
          for \(\alpha\) and \(\beta\) while maintaining invariance.
    \item Obtain an ODE in terms of \(f\left( \eta \right)\) by
          substituting the similarity solution into the PDE.
    \item Obtain boundary conditions in terms of \(\eta\) using the
          same transformation.
    \item Solve the ODE to find \(f\left( \eta \right)\).
    \item Transform back to the original variables to find a solution
          to the PDE.
\end{enumerate}
\section{Travelling Wave Solutions}
Another type of solution is the \textit{travelling wave solution},
where the solution to a PDE appears to move at a constant velocity when
after a long period of time. This is similar to a steady state solution
where the solution does not change after a long period of time. In this
problem, we assume solutions of the form
\begin{equation*}
    u\left( x,\: t \right) = f\left( z \right), \quad z = x - c t,
\end{equation*}
where \(c\) is the speed at which the solution travels. The travelling
wave speed is often determined using analysis, boundary conditions,
numerical methods, or physical constraints. This method is only
applicable to PDEs that are invariant under translations to both
independent variables. As such, the PDE must not contain any explicit
dependence on \(x\) or \(t\).
\part{Method of Characteristics}
The method of characteristics is a technique used to solve nonlinear
PDEs by reducing them to a system of ODEs. It considers
parametrisations of the solution through a set of curves in the
solution space along which the solution is constant.
\section{First Order PDEs}
\subsection{Linearity}
Consider the general first order PDE in two variables:
\begin{equation*}
    a \pdv{u}{x} + b \pdv{u}{y} = c.
\end{equation*}
This PDE has four classifications based on the value of the
variables \(a\), \(b\), and \(f\):
\begin{itemize}
    \item \textbf{Linear}:
          \begin{equation*}
              a\left( x,\: y \right) \pdv{u}{x} + b\left( x,\: y \right) \pdv{u}{y} = c\left( x,\: y \right).
          \end{equation*}
    \item \textbf{Semi-Linear}:
          \begin{equation*}
              a\left( x,\: y \right) \pdv{u}{x} + b\left( x,\: y \right) \pdv{u}{y} = c\left( x,\: y,\: u \right).
          \end{equation*}
    \item \textbf{Quasi-Linear}:
          \begin{equation*}
              a\left( x,\: y,\: u \right) \pdv{u}{x} + b\left( x,\: y,\: u \right) \pdv{u}{y} = c\left( x,\: y,\: u \right).
          \end{equation*}
    \item \textbf{Nonlinear}: Otherwise.
\end{itemize}
\subsection{Solution Method}
Given a quasi-linear PDE of the form:
\begin{equation*}
    a\left( x,\: y,\: u \right) \pdv{u}{x} + b\left( x,\: y,\: u \right) \pdv{u}{y} = c\left( x,\: y,\: u \right)
\end{equation*}
with the initial condition \(u\left( x_0,\: y_0 \right) = u_0\left( x \right)\),
let us consider the surface \(u = u\left( x,\: y \right)\), which has
the normal vector:
\begin{equation*}
    \symbf{n} =
    \begin{bmatrix}
        \pdv{u}{x} \\
        \pdv{u}{y} \\
        -1
    \end{bmatrix}
    .
\end{equation*}
We can find a tangent vector to \(\symbf{n}\) using the dot product:
\begin{equation*}
    \begin{bmatrix}
        a\left( x,\: y,\: u \right) \\
        b\left( x,\: y,\: u \right) \\
        c\left( x,\: y,\: u \right)
    \end{bmatrix}
    \cdot
    \symbf{n}
    = 0.
\end{equation*}
We will call this vector \(\symbf{v}\):
\begin{equation*}
    \symbf{v} =
    \begin{bmatrix}
        a\left( x,\: y,\: u \right) \\
        b\left( x,\: y,\: u \right) \\
        c\left( x,\: y,\: u \right)
    \end{bmatrix}
    .
\end{equation*}
Let us now consider a parametric curve defined as:
\begin{equation*}
    \symbf{r}\left( s \right) =
    \begin{bmatrix}
        x\left( s \right) \\
        y\left( s \right) \\
        u\left( s \right)
    \end{bmatrix}
    ,
\end{equation*}
and let us impose that \(\symbf{r}'\left( s \right) = \symbf{v}\), so
that the curve lies on the solution surface. This gives us the following
system of equations:
\begin{equation*}
    \symbf{r}'\left( s \right) = \symbf{v} \iff
    \begin{cases}
        \displaystyle\pdv{x}{s} = a\left( x,\: y,\: u \right), \\
        \displaystyle\pdv{y}{s} = b\left( x,\: y,\: u \right), \\
        \displaystyle\pdv{u}{s} = c\left( x,\: y,\: u \right).
    \end{cases}
\end{equation*}
These parametric curves are found through integration and we can solve
all integration constants \(x_0\), \(y_0\), and \(u_0\) using the
parametric initial condition:
\begin{equation*}
    u\left( 0 \right) = u_0\left( \xi \right) \quad \text{on} \quad x\left( 0 \right) = x_0\left( \xi \right), \quad y\left( 0 \right) = y_0\left(\xi \right).
\end{equation*}
with the parameter \(\xi\)\footnote{\(\xi\) is introduced to parametrise
the variable \(x\) in the initial condition.}. The resulting parametric
equations \(x\left( s \right)\) and \(y\left( s \right)\) form
\textit{characteristic curves} of the PDE, and we can find the solution
\(u\left( x,\: y \right)\) by eliminating \(s\) and \(\xi\) from the
three parametric equations.
\section{System of First Order PDEs}
Consider an \(n \times n\) coupled system of first order PDEs:
\begin{equation*}
    \symbf{A} \pdv{\symbf{u}}{x} + \symbf{B} \pdv{\symbf{u}}{y} = \symbf{c},
\end{equation*}
where \(\symbf{A} = \symbf{A}\left( x,\: y,\: \symbf{u} \right)\) and
\(\symbf{B} = \symbf{B}\left( x,\: y,\: \symbf{u} \right)\) are
\(n \times n\) matrix functions, and
\(\symbf{c} = \symbf{c}\left( x,\: y,\: \symbf{u} \right)\) is an
\(n \times 1\) vector function.
In this case, we want to find characteristic directions \(\symbf{v}\)
that will allow us to decouple the system. We will try to decompose
\(\symbf{A}\) and \(\symbf{B}\) into a diagonal form by assuming the
following relationship holds for some \(\symbf{m}\):
\begin{equation*}
    \symbf{v}^\top \left( \symbf{A} \pdv{\symbf{u}}{x} + \symbf{B} \pdv{\symbf{u}}{y} \right) = \symbf{m}^\top \left( \alpha \pdv{\symbf{u}}{x} + \beta \pdv{\symbf{u}}{y} \right),
\end{equation*}
where \(\symbf{v}\) and \(\symbf{m}\) are \(n \times 1\) vector
functions, and \(\alpha = \odv{x}{s}\) and \(\beta = \odv{y}{s}\) are
parametric functions. For this to hold, we must have:
\begin{equation*}
    \symbf{v}^\top \symbf{A} = \symbf{m}^\top \alpha \quad \text{and} \quad \symbf{v}^\top \symbf{B} = \symbf{m}^\top \beta.
\end{equation*}
By eliminating \(\symbf{m}^\top\), we can find a relationship between
\(\symbf{A}\) and \(\symbf{B}\):
\begin{align*}
    \frac{1}{\alpha} \symbf{v}^\top \symbf{A} & = \frac{1}{\beta} \symbf{v}^\top \symbf{B}      \\
    \symbf{v}^\top \symbf{A}                  & = \frac{\alpha}{\beta} \symbf{v}^\top \symbf{B} \\
    \symbf{v}^\top \symbf{A}                  & = \lambda \symbf{v}^\top \symbf{B}.
\end{align*}
This is precisely the left-generalised eigenvalue problem for the matrix
pair \(\left( \symbf{A},\: \symbf{B} \right)\) where the eigenvalues
\(\lambda_i\) are found by solving the characteristic equation:
\begin{equation*}
    \det{\left( \symbf{A} - \lambda \symbf{B} \right)} = 0.
\end{equation*}
Assuming a diagonalisable system, we have the following matrix
decomposition:
\begin{equation*}
    \symbf{A} = \symbf{B} \symbf{V} \symbf{\Lambda} \symbf{V}^{-1},
\end{equation*}
where \(\symbf{V}\) is the matrix of eigenvectors and \(\symbf{\Lambda}\)
is the diagonal matrix of eigenvalues. Let us now define a new variable
\(\symbf{w} = \symbf{V}^{-1} \symbf{u}\) to decouple the system:
\begin{align*}
    \symbf{A} \pdv{\symbf{u}}{x} + \symbf{B} \pdv{\symbf{u}}{y}                                    & = \symbf{c}                                \\
    \symbf{V}^{-1} \symbf{B}^{-1} \symbf{A} \pdv{\symbf{u}}{x} + \symbf{V}^{-1} \pdv{\symbf{u}}{y} & = \symbf{V}^{-1} \symbf{B}^{-1} \symbf{c}  \\
    \symbf{\Lambda} \pdv{\symbf{w}}{x} + \pdv{\symbf{w}}{y}                                        & = \symbf{V}^{-1} \symbf{B}^{-1} \symbf{c}.
\end{align*}
As \(\symbf{\Lambda}\) is diagonal, we can form a system of \(n\) first
order PDEs which can be solved independently to find
\(w_i\left( x,\: y \right)\). Upon finding these solutions, we can
transform back to the original variable \(\symbf{u}\) using
\(\symbf{u} = \symbf{V} \symbf{w}\).
\section{Second Order PDEs}
Consider the general linear second order PDE in two variables:
\begin{equation*}
    a \pdv[order=2]{u}{x} + 2b \pdv{u}{x,y} + c \pdv[order=2]{u}{y} + d \pdv{u}{x} + e \pdv{u}{y} = f.
\end{equation*}
with the initial conditions \(u\left( x_0,\: y_0 \right) = f\left( x \right)\)
and \(u_y\left( x_0,\: y_0 \right) = g\left( x \right)\) where variables
\(a\) through \(f\) are functions of \(x\) and \(y\) only. We will begin
by factoring this PDE into it's two first derivatives of \(u\),
\(u_x = \pdv{u}{x}\) and \(u_y = \pdv{u}{y}\), to convert the second
order PDE into a system of first order PDEs:
\begin{align*}
    a \pdv[order=2]{u}{x} + 2b \pdv{u}{y,x} + c \pdv[order=2]{u}{y} + d \pdv{u}{x} + e \pdv{u}{y} & = f                  \\
    a \pdv{u_x}{x} + 2b \pdv{u_x}{y} + c \pdv{u_y}{y} + d u_x + e u_y                             & = f                  \\
    a \pdv{u_x}{x} + 2b \pdv{u_x}{y} + c \pdv{u_y}{y}                                             & = f - d u_x - e u_y,
\end{align*}
where we assume Schwarz's theorem holds:
\begin{equation*}
    \pdv{u}{y,x} = \pdv{u}{x,y} \iff \pdv{u_x}{y} = \pdv{u_y}{x}.
\end{equation*}
We can then form the following system of first order PDEs:
\begin{equation*}
    \begin{bmatrix}
        a & 0 \\
        0 & 1
    \end{bmatrix}
    \pdv{}{x}
    \begin{bmatrix}
        u_x \\
        u_y
    \end{bmatrix}
    +
    \begin{bmatrix}
        2b & c \\
        -1 & 0
    \end{bmatrix}
    \pdv{}{y}
    \begin{bmatrix}
        u_x \\
        u_y
    \end{bmatrix}
    =
    \begin{bmatrix}
        f - d u_x - e u_y \\
        0
    \end{bmatrix}
    .
\end{equation*}
This system can now be solved using the process described in the
previous section, where the initial condition for \(u_x\) is simply
\(u_x\left( x_0,\: y_0 \right) = f'\left( x \right)\).
\subsection{Classification of Second Order PDEs}
The eigenvalue problem for this system is given by:
\begin{equation*}
    \begin{vmatrix}
        a - 2b \lambda & -c       \\
        1              & -\lambda
    \end{vmatrix}
    = a \lambda^2 - 2b \lambda + c = 0,
\end{equation*}
and the eigenvalues are given by:
\begin{equation*}
    \lambda = \frac{2b \pm \sqrt{4b^2 - 4ac}}{2a} = \frac{b \pm \sqrt{b^2 - ac}}{a}.
\end{equation*}
We can classify this second order PDE into 3 types based on the
discriminant \(b^2 - ac\):
\begin{itemize}
    \item \textbf{Hyperbolic}: (two real distinct eigenvalues)
          \begin{equation*}
              b^2 - ac > 0.
          \end{equation*}
    \item \textbf{Parabolic}: (two real equal eigenvalues)
          \begin{equation*}
              b^2 - ac = 0.
          \end{equation*}
    \item \textbf{Elliptic}: (two complex eigenvalues)
          \begin{equation*}
              b^2 - ac < 0.
          \end{equation*}
\end{itemize}
\end{document}
